\section{Our Model}

Here we present a first draft of our model for a privacy-preserving
event-ambivalent notification scheme~\footnote{Design and presentation of this
model inspired by Section 3.1 of~\cite{CK:eurocrypt20}.}.\par

Our model requires that the event server is not able to record the sender of
incoming messages. This can be achieved in practice either by assumption or
requiring the client to do some sort of packet spoofing or use a communication
protocol like Tor to blind the source.

Informally, an event-ambivalent notification scheme is a protocol between a
client, an event server, and an alert server. The protocol consists of the
following steps:\par

\begin{enumerate}
  \item The client uses the \texttt{Setup} algorithm to generate a secret key
    \texttt{sk}.
  \item Upon an event, the client uses \texttt{Identify} to generate an
    \texttt{tag} and a \texttt{hint} given \texttt{sk} and the event. The client
    sends the \texttt{tag} and the event to the event server and sends the
    \texttt{hint} to the alert server.
  \item When the event server wants to send out an alert based on a specific
    event, it runs \texttt{Process} on the event \texttt{tag} and sends the 
    resulting \texttt{badge} to the alert server.
  \item Upon receiving an \texttt{badge} and the \texttt{hint} it stored, the
    alert server runs \texttt{Label} to gain an address \texttt{addr} that it
    then places the alert in the mailbox labeled \texttt{addr}.
  \item The client runs \texttt{Address} using \texttt{sk} to check receive a
    set of mailboxes \texttt{addrs} that they can periodically check for
    messages.
\end{enumerate}

Provided the two servers do not collude, a privacy-preserving event-ambivalent
notification scheme should ensure that:\par
\begin{enumerate}
  \item The event server shouldn't be able to link different events stored by the
    same client together.
  \item The alert server shouldn't learn anything from an identifier regarding the
    event it is tied to or what client generated it.
  \item The client shouldn't be able to link a message in a mailbox back to a
    specific event.
  \item The servers and any other clients shouldn't be able to link any mailboxes
    to a specific user.
  \item Only a client who submitted an event can then check the resulting mailbox.
\end{enumerate}

A rough draft of the formal definition of such a scheme:\par

\begin{define}
\label{define:model}
  
  \emph{A privacy-preserving event-ambivalent notification scheme is a tuple}\\
  $\Pi = \mathtt{(Setup, Identify, Process, Label, Address)}$ \emph{of efficient
  algorithms:}
  
  \begin{itemize}
    
    \item $\mathtt{Setup}(1^\lambda) \to \mathtt{sk}$:

    \item $\mathtt{Identify}(\mathtt{sk, event}) \to (\mathtt{tag, hint}$:

    \item $\mathtt{Process}(\mathtt{tag}) \to \mathtt{badge}$:
    
    \item $\mathtt{Label}(\mathtt{badge, hint}) \to \mathtt{addr}$:

    \item $\mathtt{Address}(\mathtt{sk}) \to \mathtt{addrs}$:
  
  \end{itemize}

\end{define}

Furthermore, $\Pi$ must satisfy the following properties:

\textbf{Correctness.}

\[
  \textrm{Pr}\left[ 
    \mathtt{Label}(\mathtt{alert, hint}) \in \mathtt{addr} : 
    \begin{tabular}{r}
      $\mathtt{sk} \gets \mathtt{Setup}\left(1^\lambda\right)$\\
      $(\mathtt{id, hint}) \gets \mathtt{Identify}\left(\mathtt{sk, event}\right)$\\
      $\mathtt{alert} \gets \mathtt{Process}\left(\mathtt{id}\right)$\\
      $\mathtt{addr} \gets \mathtt{Address}\left(\mathtt{sk}\right)$\\
    \end{tabular} 
  \right] = 1
\]

\textbf{Security.}



